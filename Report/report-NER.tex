%\documentclass[journal, a4paper]{IEEEtran}
\documentclass{article}
\usepackage{graphicx}   
\usepackage{subcaption}
\usepackage[section]{placeins}
\usepackage[utf8]{inputenc}
\usepackage[margin=2.5cm]{geometry}
\usepackage{listings}
\usepackage{color}

\usepackage{blindtext}

\usepackage{pgfplots}

\definecolor{dkgreen}{rgb}{0,0.6,0}
\definecolor{gray}{rgb}{0.5,0.5,0.5}
\definecolor{mauve}{rgb}{0.58,0,0.82}
\usepackage{tikz}
\usetikzlibrary{plotmarks}
% The data files, written on the first run.

\begin{document}

\section{Named Entity Recognition}

For the name entity recognition task, the code provided by the course was upgraded by adding features and then by modifying the model.

\subsection{Constrained}

Multiple features were designed. We first started by designing the features in the chapter 21 of Jurafsky and Martin’s book:
\begin{itemize}  
	\item identity of wit
	\item identity of neighboring words
	\item part of speech of wi
	\item part of speech of neighboring words
	\item wi contains a particular prefix (from all prefixes of length less than 4)
	\item wi contains a particular suffix (from all suffixes of length less than 4)
	\item wi is all upper case
	\item word shape of wi
	\item word shape of neighboring words
	\item short word shape of wi
	\item short word shape of neighboring words
	\item presence of hyphen 
\end{itemize}

The features "base-phrase syntactic chunk label of wi and neighboring words" was not implemented.

And the last features mentioned in the book is the use of a gazetteer. For this, we searched online had to search online and found one at http://download.geonames.org/, a gazetteer that was in the list of gazetteer from wikipedia. This gazetteer provided us with a list of names and aliases for places in the whole word, but also with categories for this places. The use of a gazetteer improved our score and we were surprised that the score didn't went even higher. 
\newline
Other features computed where:\newline
The word in lowercase.\newline 
How many time the word repeated in the previous 50 words or next 50 words in the corpus. This is because analyzing the corpus we saw that the proper names were repeated often.\newline
Combination features, were the shape of the word is concatenated with the shape of the neighbor.\newline
A features telling if the word is the begining or the end of the sentence\newline
Also, we changed the number of neighbors to see the 3 last neighbors and the next three neighbors when computing all the previous features.\newline

the adaboost classifier and  the random forest classifiers were tried but both yield worse results.

\subsection{Unconstrained}
For the unconstrained results, a different model was tried. this model is CRF from sklearn\_crfsuite. This model improved our score greatly. This is because in the perceptron model, we could not find a way to tell the model that after predicting for example B-ORG, we couldn't have I-ORG. Because everything was predicted in the .predict method of the classifier, we could not use the previous prediction as features. But the CRF has transitional probabilities and this allow for those errors to be prevented.

\subsection{Conclusion}
Overall we found that every features was really important, and that gazetteer, even if they improve the score, are not magical solutions that solve the problem automatically. After reading many papers, many interesting methods were found, like stacking models or using more complex features. this was not done because of a time constraint but it would have been really interesting to test them. 

\end{document}	
